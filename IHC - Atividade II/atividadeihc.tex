% abtex2-modelo-artigo.tex, v-1.9.2 laurocesar
% Copyright 2012-2014 by abnTeX2 group at http://abntex2.googlecode.com/ 
%

% ------------------------------------------------------------------------
% ------------------------------------------------------------------------
% abnTeX2: Modelo de Artigo Acadêmico em conformidade com
% ABNT NBR 6022:2003: Informação e documentação - Artigo em publicação 
% periódica científica impressa - Apresentação
% ------------------------------------------------------------------------
% ------------------------------------------------------------------------


\documentclass[
	% -- opções da classe memoir --
	article,			% indica que é um artigo acadêmico
	11pt,				% tamanho da fonte
	oneside,			% para impressão apenas no verso. Oposto a twoside
	a4paper,			% tamanho do papel. 
	% -- opções da classe abntex2 --
	%chapter=TITLE,		% títulos de capítulos convertidos em letras maiúsculas
	%section=TITLE,		% títulos de seções convertidos em letras maiúsculas
	%subsection=TITLE,	% títulos de subseções convertidos em letras maiúsculas
	%subsubsection=TITLE % títulos de subsubseções convertidos em letras maiúsculas
	% -- opções do pacote babel --
	english,			% idioma adicional para hifenização
	brazil,				% o último idioma é o principal do documento
	sumario=tradicional
	]{abntex2}


% ---
% PACOTES
% ---

% ---
% Pacotes fundamentais 
% ---
\usepackage{lmodern}			% Usa a fonte Latin Modern
\usepackage[T1]{fontenc}		% Selecao de codigos de fonte.
\usepackage[utf8]{inputenc}		% Codificacao do documento (conversão automática dos acentos)
\usepackage{indentfirst}		% Indenta o primeiro parágrafo de cada seção.
\usepackage{nomencl} 			% Lista de simbolos
\usepackage{color}				% Controle das cores
\usepackage{graphicx}			% Inclusão de gráficos
\usepackage{microtype} 			% para melhorias de justificação
% ---
		
% ---
% Pacotes adicionais, usados apenas no âmbito do Modelo Canônico do abnteX2
% ---
\usepackage{lipsum}				% para geração de dummy text
% ---
		
% ---
% Pacotes de citações
% ---
\usepackage[brazilian,hyperpageref]{backref}	 % Paginas com as citações na bibl
\usepackage[alf]{abntex2cite}	% Citações padrão ABNT
% ---

% ---
% Configurações do pacote backref
% Usado sem a opção hyperpageref de backref
\renewcommand{\backrefpagesname}{Citado na(s) página(s):~}
% Texto padrão antes do número das páginas
\renewcommand{\backref}{}
% Define os textos da citação
\renewcommand*{\backrefalt}[4]{
	\ifcase #1 %
		Nenhuma citação no texto.%
	\or
		Citado na página #2.%
	\else
		Citado #1 vezes nas páginas #2.%
	\fi}%
% ---

% ---
% Informações de dados para CAPA e FOLHA DE ROSTO
% ---
\titulo{Atividade II de Interação Humano-Computador}
\autor{
	Antônio F. de Carvalho$^{1}$, Diego F. Sousa Lima$^{1}$\\
	Diego de S. Vasconcelos$^{1}$, Marcio Silvano$^{1}$
	\\
	$^{1}$Curso Bacharelado em Sistemas de Informação -- Universidade Federal do Piauí\\
	\texttt{\{antonio007023, diegofernando5672\}@gmail.com}\\
	\texttt{\{diegosousa.33, marcinho944\}@hotmail.com}
}


% ---

% ---
% Configurações de aparência do PDF final

% alterando o aspecto da cor azul
\definecolor{blue}{RGB}{41,5,195}

% informações do PDF
\makeatletter
\hypersetup{
     	%pagebackref=true,
		pdftitle={\@title}, 
		pdfauthor={\@author},
    	pdfsubject={Modelo de artigo científico com abnTeX2},
	    pdfcreator={LaTeX with abnTeX2},
		pdfkeywords={abnt}{latex}{abntex}{abntex2}{atigo científico}, 
		colorlinks=true,       		% false: boxed links; true: colored links
    	linkcolor=blue,          	% color of internal links
    	citecolor=blue,        		% color of links to bibliography
    	filecolor=magenta,      		% color of file links
		urlcolor=blue,
		bookmarksdepth=4
}
\makeatother
% --- 

% ---
% compila o indice
% ---
\makeindex
% ---

% ---
% Altera as margens padrões
% ---
\setlrmarginsandblock{3cm}{3cm}{*}
\setulmarginsandblock{3cm}{3cm}{*}
\checkandfixthelayout
% ---

% --- 
% Espaçamentos entre linhas e parágrafos 
% --- 

% O tamanho do parágrafo é dado por:
\setlength{\parindent}{1.3cm}

% Controle do espaçamento entre um parágrafo e outro:
\setlength{\parskip}{0.2cm}  % tente também \onelineskip

% Espaçamento simples
\SingleSpacing

% ----
% Início do documento
% ----
\begin{document}

% Retira espaço extra obsoleto entre as frases.
\frenchspacing 

% ----------------------------------------------------------
% ELEMENTOS PRÉ-TEXTUAIS
% ----------------------------------------------------------


% página de titulo
\maketitle


% resumo em português
% ]  				% FIM DE ARTIGO EM DUAS COLUNAS
% ---

% ----------------------------------------------------------
% ELEMENTOS TEXTUAIS
% ----------------------------------------------------------
\textual

% ----------------------------------------------------------
% Introdução
% ----------------------------------------------------------

Todas as questões foram respondidas baseadas em \citeonline{barbosa2010interaccao}.\\

\textbf{1) \textit{Fatores de Usabilidade} . Identifique quais fatores de usabilidade deveriam ser privilegiados nos seguintes casos}

\begin{itemize}
	\item \textbf{um sistema para gestão dos documentos produzidos  e consumidos por uma organização;}
	\item \textbf{um quiosque de informações em uma livraria;}
	\item \textbf{um caixa eletrônico;}
	\item \textbf{um sistema \textit{Web} para fornecer os resultados de exames de saúde a pacientes e seus médicos;}
	\item \textbf{um jogo educacional de simulação de fenômenos físicos (e.g., deslocamento, aceleração e atrito).}
\end{itemize}

Para um sistema para gestão dos documentos produzidos e consumidos para uma organização, o fator de segurança no uso deve ser levado em consideração juntamente com um pouco de facilidade de aprendizado.

Facilidade de aprendizado, facilidade de recordação, eficiência e satisfação do usuário com certeza devem fazer parte do escopo de fatores envolvidos em um quiosque de informações em uma livraria.

Um caixa eletrônico preza pela segurança da informação, e por isso o maior fator de usabilidade deve ser a segurança no uso. Outros fatores também podem ser interessantes como, facilidade de recordação e eficiência.

Num sistema \textit{Web} para fornecer os resultados de exames de saúde a pacientes e seus médicos, os fatores de usabilidade podem se resumir a: eficiência e facilidade de aprendizado.

Por último, um jogo educacional de simulação de fenômenos físicos pode se beneficiar dos fatores de usabilidade: facilidade de aprendizado e facilidade de recordação.

\textbf{2) \textit{Acessibilidade}. Cite exemplos de sistemas interativos para os quais a acessibilidade beneficiaria seus usuários em certas situações. Discuta os benefícios da acessibilidade nesses sistemas para os usuários e para a organização responsável pelo sistema.}

Sistemas de compra pela \textit{Internet}. Onde possamos fazer compra sem precisar sair de casa.

Sistemas de pagamentos de conta, onde o cliente pode colocas suas contas em debito em conta, Quando chegar a fartura do cartao ocorre o desconto automaticamente.


\textbf{3) \textit{Comunicabilidade}. Na interface do \textit{Microsoft Powerpoint®\footnote{\href{https://products.office.com/powerpoint}{https://products.office.com/powerpoint}}} ou posterior, analise os signos correspondentes ao uso da caneta (\textit{ink}). Tente identificar a visão do designer sobre para que serve esse recurso e como ele deve ser utilizado. Compare a edição de ilustrações utilizando a caneta e utilizando formas geométricas predefinidas (e.g., criação, modificação, seleção, agrupamento e deslocamento das ilustrações)}

No \textit{PowerPoint 2013}, o Designer ao criar a opção \textit{"Novo Slide"} deixa bem claro ao usuário que ao clicar naquela opção será criado um novo \textit{slide} e ao colocar o cursor do \textit{mouse} sobre a opção é exibido uma mensagem informando ao usuário que será inserido um novo \textit{slide} na área de apresentação e ainda informa o atalho para que da próxima vez o mesmo possa criar um \textit{slide} sem precisar clicar de novo no botão. O \textit{Designer} preocupou-se também em criar juntamente com a opção uma outra opção para que o usuário possa definir o \textit{layout} de seus slides informando o tipo de \textit{layout} e mostrando uma imagem para o usuário visualizar como funciona o layout escolhido.

\textbf{4) \textit{Critérios de qualidade de uso}. Escolha alguns sistemas interativos a que você tenha acesso e que possa utilizar. Inspecione sua interface para analisar usabilidade, experiência do usuário, acessibilidade e comunicabilidade, considerando diferentes perfis de usuário:}

\begin{itemize}
	\item \textbf{um usuário que está utilizando o sistema pela primeira vez;}
	\item \textbf{um usuário que utiliza o sistema diariamente;}
	\item \textbf{um usuário que enxerga com dificuldade;}
	\item \textbf{um usuário com baixo grau de instrução ou analfabeto funcional;}
	\item \textbf{um usuário que tem baixo poder de concentração;}
	\item \textbf{um usuário que realiza diversas atividades ao mesmo tempo e é interrompido com frequência;}
	\item \textbf{um usuário que realiza uma tarefa longa, que precisa ser suspendida no final do dia e retomada no dia seguinte.}	
\end{itemize}


Sistema Interativo: \textit{WhatsApp} (\textit{Android})\footnote{\href{https://www.whatsapp.com/}{https://www.whatsapp.com/}}.

\begin{enumerate}
	\item Na primeira utilização, o usuário é obrigado a inserir seu número de celular para ter acesso ao aplicativo Após isso, o aplicativo carregará sua lista de contatos a partir de seu dispositivo e informará quais possuem \textit{WhatsApp}. A partir desse ponto basta clicar no usuário escolhido para a interface mudar para um sistema de chat padrão. Todo o uso é intuitivo e um novo usuário não encontrará dificuldades para o uso principal, conversar.
	
	\item Um usuário que utiliza diariamente, logo ao abrir o aplicativo, estará na tela com seus contatos que possuem \textit{WhatsApp}. Possivelmente, ele já explorou outras funções, estas estão disponíveis em um único botão que dá opções de criação de novo grupo, alteração de status, configuração do aplicativo etc. Na opção de configuração, haverá ícones com seus respectivos textos, sendo o primeiro deles "\textit{Ajuda}", representado pelo ícone de interrogação, ao acessá-lo o usuário terá a opção de acessar um \textit{FAQ} de perguntas e respostas ou até enviar \textit{e-mail} para os gerenciados do sistema. Pontos positivos para os usuários diários seriam a facilidade de recordação e a satisfação do usuário.
	
	\item No quesito acessibilidade, o programa pode aparentar ser falho. As cores utilizadas são claras, por exemplo, o ícone de confirmação de envio da mensagem pode se tornar de difícil visualização. Em compensação, as cores são bem utilizadas em outros pontos do aplicativo, diferenciando cada usuário. Por outro lado, é certo que algumas funções de acessibilidade são disponibilizadas pelo sistema \textit{Android}, portanto, ao configurá-las no seu \textit{smartphone}, que vale destacar, não possui fácil aprendizagem, o aplicativo \textit{WhatsApp} poderá ler qualquer mensagem, redigi-las por voz, alterar o contraste, tamanho de texto etc.
	
	\item Nesse caso, o usuário pode apresentará dificuldades de uso no próprio uso do \textit{smartphone}, consequentemente no uso do aplicativo. Provavelmente, somente as funções mais destacadas na interface seriam utilizadas, que seria o envio de mensagens aos contatos que aparecem na tela de início. O uso do \textit{smartphone} seria muitíssimo mais complicado que o uso do aplicativo.
	
	\item Como a interface é polida e com poucos botões, não acredito que um usuário com baixo poder de concentração encontrará problemas, pois as funções do aplicativos são simples, somente comunicação de texto, imagens ou voz.
	
	\item O aplicativo salvará o texto não enviado, mesmo que o usuário feche o aplicativo por descuido ou outras ocupações. Podendo retornar ao aplicativo e continuar sua mensagem ou leitura de mensagens recebidas do ponto que parou. Cabe um detalhe, se o usuário permanecer muito tempo sem utilizar o \textit{WhatsApp}, o sistema \textit{Android} encarregará de fechar o aplicativo e o texto digitado, mas não enviado, seria totalmente perdido.
	
	\item Todas as mensagens estariam salvas e disponíveis para a leitura em qualquer horário, mas o problema apontado no item anterior persistiria. Mensagens digitadas, mas não enviadas seriam perdidas. No caso, é uma limitação do sistema operacional (\textit{Android}), o qual para manter o texto salvos mesmo após longas horas de inatividade demandariam uso contínuo do sistema e maior consumo da bateria, prejudicando o uso em geral do aparelho.
\end{enumerate}

% ----------------------------------------------------------
% Referências bibliográficas
% ----------------------------------------------------------
\bibliography{abntex2-modelo-references}

\end{document}
