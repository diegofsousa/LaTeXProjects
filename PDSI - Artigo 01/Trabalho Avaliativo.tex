\documentclass[12pt]{article}

\usepackage{sbc-template}

\usepackage{graphicx,url}

\usepackage[brazil]{babel}   
%\usepackage[latin1]{inputenc}  
\usepackage[utf8]{inputenc}  
% UTF-8 encoding is recommended by ShareLaTex

     
\sloppy

\title{FixCodeApp:\\ Um aplicativo para desenvolvedores}

\author{Diego F. Sousa Lima\inst{1},Bruno L. Alcântara\inst{1} }


\address{Curso Bacharelado em Sistemas de Informação -- Universidade Federal do Piauí
  (UFPI)\\
  Junco, Picos -- PI, 64600-000 -- Brazil
  \email{\{diegofernando5672,brunolopes.ips\}@gmail.com}
}

\begin{document} 

\maketitle


\section{O que é FixCodeApp}

\textbf{FixCodeApp} irá se comportar como aplicativo híbrido complementar a plataforma web, esta que se comporta como um meio de auxílio aos desenvolvedores e programadores permitindo o usuário poder, tanto sanar dúvidas com relação à algoritmos, como também expor projetos e ideias com possibilidade de fazer o upload de arquivo. A plataforma irá gerenciar os dados de acordo com Fixies (Perguntas) e Posts (Exibição de ideias) separados por linguagens de programação,  autor, data e etc.

 Os usuários poderão escolher até 5 habilidades (linguagens) pessoais para seu perfil. O sistema contará também com um sistema de
seguidores a fim de que um usuário receba atualizações de outro usuário seguido na timeline.


A problemática se deu no princípio de que sempre quando temos algum tipo de dúvida, interrogação na área da TI, e mais especificamente na parte do desenvolvimento, temos algumas opções pela internet. São os fóruns. A plataforma \textbf{FixCode} demonstra em sua essência um meio mais interativo dos que existem nos fóruns hoje em dia.

 Além disso a fórmula da plataforma consiste em que um usuário tenha a resposta para sua dúvida o mais rápido possível.
 
 
A principal estratégia para a busca de uma maior produtividade para o usuário é o modelo da plataforma que é construído com o comportamento de uma rede social. O aplicativo vem para consolidar a produtividade e aumentar a interatividade, uma vez que os recursos web estarão na palma da mão do usuário.


Atualmente, o FixCode é uma plataforma web escrita em basicamente em \textbf{Python}. A iniciativa nasceu em meados de agosto do ano de 2016 quando um dos desenvolvedores da plataforma comentou a respeito de uma ferramenta que unisse os desenvolvedores e programadores a fim de que pudessem compartilhar conhecimento. 
A versão final deste projeto (Web) foi apresentada no dia 25/02/2017 para a disciplina de Programação para Web II na Universidade Federal do Piauí - Campus Senhor Helvídio Nunes de Barros.

 O próximo passo, é a construção do app. Tem-se estudado usar a tecnologia híbrida para o aplicativo, afim de que possamos criar versões para as plataformas mais famosas, nas quais são: \textit{Android}, \textit{IOS} e \textit{Windows Phone}. 


\end{document}
