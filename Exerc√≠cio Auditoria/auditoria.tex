% --------------------------------------------------------------
% This is all preamble stuff that you don't have to worry about.
% Head down to where it says "Start here"
% --------------------------------------------------------------
 
\documentclass[12pt]{article}
\usepackage[utf8]{inputenc}		% Codificacao do documento (conversão automática dos acentos)
\usepackage[brazil]{babel}   

 
\usepackage[margin=1in]{geometry} 
\usepackage{amsmath,amsthm,amssymb}
 
\newcommand{\N}{\mathbb{N}}
\newcommand{\Z}{\mathbb{Z}}
 
\newenvironment{theorem}[2][Theorem]{\begin{trivlist}
\item[\hskip \labelsep {\bfseries #1}\hskip \labelsep {\bfseries #2.}]}{\end{trivlist}}
\newenvironment{lemma}[2][Lemma]{\begin{trivlist}
\item[\hskip \labelsep {\bfseries #1}\hskip \labelsep {\bfseries #2.}]}{\end{trivlist}}
\newenvironment{exercise}[2][Exercise]{\begin{trivlist}
\item[\hskip \labelsep {\bfseries #1}\hskip \labelsep {\bfseries #2.}]}{\end{trivlist}}
\newenvironment{reflection}[2][Reflection]{\begin{trivlist}
\item[\hskip \labelsep {\bfseries #1}\hskip \labelsep {\bfseries #2.}]}{\end{trivlist}}
\newenvironment{proposition}[2][Proposition]{\begin{trivlist}
\item[\hskip \labelsep {\bfseries #1}\hskip \labelsep {\bfseries #2.}]}{\end{trivlist}}
\newenvironment{corollary}[2][Corollary]{\begin{trivlist}
\item[\hskip \labelsep {\bfseries #1}\hskip \labelsep {\bfseries #2.}]}{\end{trivlist}}
 
\begin{document}
 
% --------------------------------------------------------------
%                         Start here
% --------------------------------------------------------------
 
%\renewcommand{\qedsymbol}{\filledbox}
 
\title{Exercício complementar\\Auditoria e Segurança de Sistemas de Informação}%replace X with the appropriate number
\author{Diego Fernando de Sousa Lima\\Professor: Ismael de Holanda Leal\\ %replace with your name
Curso Bacharelado em Sistemas de Informação - UFPI - CSHNB} %if necessary, replace with your course title
 
\maketitle
 
\section{Primeiro capítulo}

\textbf{a) Qual é a importância da auditoria nas empresas?}
\\
Atualmente as empresas buscam melhorias no processo da gestão de qualidade de produtos e serviços, tendo em vista a demanda crescente de consumidores mais exigentes. Para garantir que tudo funcione corretamente com eficiência e eficácia é necessário aplicar o processo de auditoria e segurança da informação. Desta forma, além de certificar a qualidade na execução do serviço, a empresa vai se prevenir contra acessos externos não autorizados no sistema de informação.
\\

\textbf{b) O processo aplicação dos métodos da auditoria pode variar conforme a empresa?}
\\
Sim pois, cabe gerenciar as operações e os processos nas organizações, no intuito de garantir a execução e o correto funcionamento dos procedimentos de acordo com as regras e normas estabelecidas para empresa.
\\

\textbf{c) Podemos afirmar que os conceitos de segurança da informação são aplicados apenas nos sistemas computacionais?}
\\
Não, com a auditoria de sistemas é possível dimensionar a amplitude do trabalho da auditoria a ser desenvolvido, pois teremos a percepção exata dos setores e processos que serão abordados. 
\\

\textbf{d) Entre os conceitos aplicáveis na segurança da informação: confidencialidade, integridade e disponibilidade. Qual deles podemos destacar como o primordial?}
\\
Confidencialidade, pois o fator primordial da auditoria consiste na segurança das informações.
\\

\textbf{e) Explique a diferença entre "auditoria" e "auditoria de sistemas?"}
\\
Auditoria é a análise minuciosa das operações, já auditoria de sistemas pode gerenciar as operações e os processos nas organizações, no intuito de garantir a execução e o correto funcionamento dos procedimentos de acordo com as regras e normas estabelecidas para empresa.
\\

\textbf{f) Podemos afirmar que o auditor de sistemas é um profissional de conhecimento específico? Por quê?}
\\
O auditor de sistemas nunca deve ser visto como um profissional isolado, pelo fato de trabalhar diretamente com ferramentas voltadas as tecnologias.  O auditor de sistema pode ser visto como membro da equipe de auditores designada para levantamento e geração de resultados, sua função e oferecer suporte em decorrência dos avanços tecnológicos inseridos nas organizações.
\\

\textbf{g) Defina quais são as atribuições de um auditor de sistema?}
\\
O auditor de sistema realiza o controle do fluxo das informações computacionais para comprovar a efetividade dos programas, garantindo desta forma a integridade, confiabilidade e confidencialidade dos processos.
O auditor também tem o papel primordial na elaboração do plano de teste (o que será testando a nível computacional), checagem de rotinas sistêmicas (verificação e validação de dados fornecidos pelos sistemas da empresa) e diagnóstico das falhas ou vulnerabilidades encontradas na gestão da tecnologia da informação. 
\\

\textbf{h) Cite alguns benefícios proporcionados pela auditoria assistida?}
\\
Redução dos custos, Menor possibilidade de erros, Agilidade para análise dos processos

\section{Segundo capítulo}

\textbf{a) Cite os tipos de auditoria e explique suas diferenças.}
\begin{itemize}
	\item Auditoria Interna: é definida como um processo da entidade, com objetivo de verificar a eficácia da mesma, ou seja, se os seus requisitos estão sendo atendidos na realização das atividades da empresa;
	\item Auditoria Externa: é o exame minucioso das atividades desenvolvidas na empresa ou departamentos. Seu objetivo é checar se as diretrizes planejadas pelo manual da qualidade, estão sendo implementadas com eficácia, em conformidade com as suas normas estabelecidas;
	\item Auditoria Articulada: É o trabalho conjunto realizado entre a auditoria interna e externa. Sua principal função é informar ao auditor externo sobre problemas já detectados na gestão da empresa. Desta forma o auditor externo fará a avaliação das ações preventivas, detectivas e corretivas aplicadas para sanar a falha.
\end{itemize}


\textbf{b) Qual é a papel desempenhado pelo comitê gestor interno?}
\\
Entre as funções do comitê, cabe destacar a criação das rotinas de trabalhos, bem como, efetuar o acompanhamento dos indicadores definidos no manual.
\\

\textbf{c) Podemos dizer que procedimentos são normas? Explique.}
\\
Dentro do contexto da auditoria a palavra Procedimento representa a descrição das funções exercidas pelas pessoas nas empresas. Podemos definir como rotina de trabalho diária. Através dessa descrição, qualquer funcionário poderá ser realocado ou substituído, pois o procedimento, norteia passo a passo como executar a tarefa para qualquer tipo de função.
\\

\textbf{d) Quais são os processos passíveis de auditoria?}
\\
Processo de auditoria interna, externa e articulada.
\\

\textbf{e) Comente a respeito da auditoria articulada e qual a sua
finalidade.}
É o trabalho conjunto realizado entre a auditoria interna e externa. Sua principal função é informar ao auditor externo sobre problemas já detectados na gestão da empresa. Desta forma o auditor externo fará a avaliação das ações preventivas, detectivas e corretivas aplicadas para sanar a falha. 
\\

\textbf{f) Comente sobre a importância do trabalho da auditoria
externa.}
\\
O auditor externo ao planejar a auditoria define junto ao comitê de auditoria interna a natureza e as áreas envolvidas para auditagem. Exemplo:
Auditorias administrativa, contábil, financeira, legalidade, sistemas, operacional e outras. A todo o momento, existe a troca de informações entre os profissionais consultores externos e internos, na busca da solu- ção que seja ideal para a instituição.

\section{Terceiro capítulo}

\textbf{a) É possível efetuar auditoria sem planejamento? Por quê? O que é uma entidade?}
\\
Não, pois o planejamento da auditoria é a primeira fase do trabalho do auditor e deve ser realizada com uma visão técnica e bem estruturada, para garantir a eficiência nas fases posteriores. 
A palavra entidade, representa uma empresa (pessoa jurídica). Na verdade, existem diversos sinônimos para definir esse termo: entidade, empresa, organização e outros.
\\

\textbf{b) Dentre as fases do planejamento, qual é a mais importante e por quê?}
\\
Conhecer a empresa, pois é o passo inicial para o planejamento da auditoria.
\\

\textbf{c) Qual é a necessidade de alocar tempo no planejamento para análise dos sistemas informatizados?}
\\
Para que a categoria (sistema de informação) seja abordada de forma adequada ao contexto geral da auditoria, será necessário checar a confiabilidade do plano de política de segurança da gestão de tecnologia da informação. Caso não haja credibilidade na política empregada, será necessário expandir o tempo de investigação do planejamento, criando métodos que possam atender a esse quesito primordial para o fluxo seguro das informações. 
\\

\textbf{d) Qual é a relação do controle interno com a tecnologia das informações?}
\\
As técnicas de controles são aplicadas tanto no nível de instituição, bem como, na gestão de TI (Tecnologia da Informação). É importante frisar que antes de existência de um sistema computacional, já existe a empresa, que por sua vez promoverá procedimentos para uma padronização de sua rotina de trabalho. 
\\

\textbf{e) Os controles preventivos, detectivos e corretivos são aplicados somente nos sistemas de informação? Comente.}
\\
Não apenas sistemas de informação. O o controle é a monitoração, fiscalização ou exame minucioso, que obedece a determinadas expectativas, normas, convenções sobre as atividades de pessoas, órgãos ou sobre produtos, a fim de não haver se desviarem das normas preestabelecidas
\\

\textbf{f) Qual é o objetivo principal do ponto de controle?}
\\
Através do ponto de controle é possível criar indicadores que vão auxiliar as possíveis mudanças ou melhorias na organização. Bem como, auxiliar o auditor, a gerar gráficos de áreas ou regiões críticas da entidade. Essas informações são essenciais na geração do relatório final da auditória
\\

\textbf{g) A conformidade verifica a execução natural do procedimento. Comente o que é procedimento?}
\\
procedimentos são rotinas de trabalhos que auxiliam na execução da rotina diária desenvolvida na empresa.

\end{document}